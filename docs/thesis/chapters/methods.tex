\chapter*{Methods}

\section*{Behaviour tree}
Behavior tree are a method to organize the decision making of a system, they have find usage particularly in A.I for
video games and chatbot. They can be considered as an extension of finite state machine [1] \marginpar{[1]:paper}. 

The tree is  composed by different type of leaf: 
\begin{itemize}
    \item Behaviour: behaviours are the leaf of the tree, they represent an action that the robot has to perform 
                     to coplete the task
    \item Composites: composites are sets of behaviour, that will suceed based on the results of the children behaviour 
\end{itemize}



In later year they have been used also in robotics as control system for various reasons: they support reactivity, 
particular important for task where human robot collaboration is involved, modularity is improved by having 
behaviour that manage one aspect of the entire task 