\chapter*{Introduction}

\section*{General introduction}
In the latest year robot has become a more and more prominent part of the work force, becoming more present in most 
sector of the industries. Robot were particularly important in the manifacturing industry as repeated movement and task
could be automated, increasing productivity and the safety of the workers, that worked with big macinery [1].
\marginpar{[1]:paper}
Industrial robot are big, and can be dangerous to human working cloes to it, that is why during the last years [2] cobot 
has been introduced. 
\marginpar{[2]:paper}
Cobot are, lighter, smaller but safer[3] than industrial robot, using force sensors and \underline{appositi} algorithm
\marginpar{[3]:paper}
they are deemed safe to work side by side with a human under the [4] sets of law 
\marginpar{[4]:define(ISO?)}
This ensured to provide the flexibility that the industrial robot missed allowing robot to be used in more specific and 
more \underline{elastic, flexible} task, where the specific can change with time, which made them unfit for industrial 
robot, where programming tend to be more complex [4] \marginpar[4: does it?].
So the robot and the human could collaborate on a task, resulting in the human doing less repetitive or dangerous tasks.

\section*{What is HRC}
HRC, which stands for human robot collaboration, is a field of robotics that studies collaborative tasks between robots 
and human.[5] \marginpar{5:difficulties in HHC paper}.



\begin{figure}[H]
    \centering
    \hspace{5 em}
    \includegraphics[width=0.75\linewidth]{figs/quack.jpg}
    \caption{}
    \label{fig:cv_tray}
\end{figure}