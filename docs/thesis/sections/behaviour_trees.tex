\subsection*{Behaviour tree}
Behavior tree are a method to organize the decision making of a system, they have find usage particularly in A.I for
video games and chatbot. They can be considered as an extension of finite state machine [1] \marginpar{[1]:paper}. 
Behaviour trees progress during execution in discrete step called tick, at each tick behaviours are executed based on
the structure and the status of the tree. When being ticked and thus executed a behaviour will return a status, that 
represent the result of the behaviour itself, this status can be \textbf{Success}, \textbf{Failure} \textbf{Running}.

The tree is  composed by different type of nodes: 
\begin{itemize}
    \item Behaviour: behaviours are the leaf of the tree, they represent an action that the robot has to perform 
                     to complete the task. (Later on this document it will be further explained how the behaviours have
                     been defined for the completation of this ???)
                     Each behaviour is 
    \item Composites: composites are control nodes and they define how a behaviour tree is traversed. This nodes have 
                      different children which can be both behaviours or composites. When ticked composi execute these
                      children, and change their status based on them and different policies:
                      \begin{itemize}
                        \item Sequence: they execute each children in order until one of them return running or failure, 
                                        or all of them return success.
                        \item Fallback: they execute each children in order unitl one of them return success or all 
                                        of them return failure.
                        \item Parallel: they execute each children in a pseudo parallel way, each children is ticked in 
                                        order, the parallel composite will suceed when at least $M$ children suceed where 
                                        $M \leq N$ and $N$ is the number of behaviour in the parallel composite
                      \end{itemize}
                      % più esperimenti necessari per questo, un tick dell composite vuol dire un tick per ogni behaviour?
                      % soprattutto se fosse così non c'è differenza tra sequence e parallel, dovrebbe essere che a un tick dell
                      % sequence corrisponde un behaviour per sequence e fallback e multipli per il parallel
\end{itemize} 
The composites nodes are what ensure scalability for behaviour trees, as you can treat composites as complex behaviours,
which can be expanded or reused. This strategy of control also ensure reactivity, especially when the tree implements
parallel nodes, where all the children are execute together.
For all these reasons in latter years behavior tree found new employement in control of robotics arm manipulation task. 
The reactivity and scalability of behaviour tree make them a great candidate for human robot collaboration task as they
let the robto be able to react to a complex agent as a human can be and can scale easily, adding nuances and additional
behaviour easily.
