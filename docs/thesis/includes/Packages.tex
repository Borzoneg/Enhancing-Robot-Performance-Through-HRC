\usepackage[total={6.5in,8.75in},top=3cm, bottom=3cm, left=2.5cm, right=3.5cm, includefoot]{geometry}

\usepackage[utf8]{inputenc}

\usepackage{graphicx}

% include pdf filer
\usepackage{pdfpages}

% Bruges til at lave subFigures
\usepackage{mwe}

\usepackage[export]{adjustbox}
\usepackage{float}

% Formler og græske symboler
\usepackage[euler]{textgreek}
\usepackage{amsmath}

% At \par flytter den rigtige længde
\usepackage{parskip}
    % \setlength{\parindent}{20pt}
    \setlength{\parskip}{1 em}

% 1,5 linjeafstand, skrifttypen New Century Schoolbook og ingen højre allignment
\renewcommand{\baselinestretch}{1.3}
\usepackage{newcent}
%\usepackage[document]{ragged2e}


%Gør at referencer og navne på figurer og tabeller er på engelsk.
\usepackage[english]{babel}

% Links bliver blå, fx mails, og man kan klikke på det.
\usepackage{hyperref}
\hypersetup{
					colorlinks, 
					linkcolor={black},
					citecolor={black},
					urlcolor={blue}
					}
					

\usepackage{imakeidx}
\makeindex
%Gør at man kan få tekst til at gå rundt om billeder.
\usepackage{wrapfig}

%Giver mulighed for flere linjer i captions mm.
\usepackage{caption}

%Gør det muligt at få subfigures, flere forskellige billeder i en figur
\usepackage{subcaption}

%giver mulighed for at bruge \textdegree{}
\usepackage{textcomp}
%giver mulighed for \degree i mathmode
\usepackage{gensymb}
%sætter sidetal til x af y
\usepackage{lastpage}
\usepackage{fancyhdr}
\pagestyle{fancy} 
\cfoot{\thepage\ of \pageref{LastPage}}

%pakke til at lave tables
\usepackage{array}
    \newcolumntype{P}[1]{>{\centering\arraybackslash}p{#1}}

% giver mulighed for at lave C++ code snippets.
\usepackage{listings}
\usepackage{xcolor}
    % \definecolor{codegreen}{rgb}{0,0.6,0}
    % \definecolor{codegray}{rgb}{0.5,0.5,0.5}
    % \definecolor{codepurple}{rgb}{0.58,0,0.82}
    % \definecolor{backcolour}{rgb}{0.95,0.95,0.92}
    % \newcommand{\mc}[2]{\multicolumn{#1}{c}{#2}}
    % \definecolor{Gray}{gray}{0.85}
    % \definecolor{toprow}{rgb}{0.52, 0.52, 0.51}
    % \definecolor{emerald}{rgb}{0.31, 0.78, 0.47}
    % \definecolor{gold(web)(golden)}{rgb}{1.0, 0.84, 0.0}
    % \definecolor{persianred}{rgb}{0.8, 0.2, 0.2}
    % \definecolor{ao(english)}{rgb}{0.0, 0.5, 0.0}
    % \newcolumntype{a}{>{\columncolor{Gray}}c}
    % \newcolumntype{b}{>{\columncolor{white}}c}
    % \newcolumntype{?}[1]{!{\vrule width #1}}
    % \definecolor{Red}{rgb}{1,0,0}

%\definecolor{backcolour}{rgb}{0.67,0.74,0.73}
\lstdefinestyle{mystyle}{
    language=C++,
    backgroundcolor=\color{backcolour},
    commentstyle=\color{codegreen},
    keywordstyle=\color{magenta},
    numberstyle=\tiny\color{codegray},
    stringstyle=\color{codepurple},
    basicstyle=\ttfamily\footnotesize,
    breakatwhitespace=false,         
    breaklines=true,                 
    captionpos=b,                    
    keepspaces=true,                 
    numbers=left,                    
    numbersep=5pt,                  
    showspaces=false,                
    showstringspaces=false,
    showtabs=false,                  
    tabsize=2
}
\lstdefinestyle{base}{
  language=C,
  emptylines=1,
  breaklines=true,
  basicstyle=\ttfamily\color{black},
  moredelim=**[is][\color{red}]{@}{@},
}

\lstset{style=mystyle}

% Kan tilføje farver til tabeller
\usepackage{colortbl}
\usepackage{titlesec}
\setcounter{secnumdepth}{4}
\titleformat{\paragraph}
{\normalfont\normalsize\bfseries}{\theparagraph}{1em}{}
\titlespacing*{\paragraph}
{0pt}{3.25ex plus 1ex minus .2ex}{1.5ex plus .2ex}

%Kan bruges til at indsætte quotes.
\usepackage{csquotes}

%Centrerer titler i tabeller.
\usepackage{booktabs}

%bruges til at gøre alle kolonner samme størrelse
\usepackage{tabularx}

%For at få nummer på references afsnit
\usepackage[numbib]{tocbibind}

\usepackage{hhline}

%For at kunne lave pseudocode i latex
\usepackage{algorithm}
\usepackage{algorithmic}